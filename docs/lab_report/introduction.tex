\section{Introduction}

\plant[e]{Cs}\index{Cannabis sativa L.} is a plant of significant economic and medicinal interest, cultivated for its various applications including fiber, seed oil, and pharmacologically active compounds such as cannabinoids and terpenes. The growth and development of cannabis plants can be influenced by numerous environmental factors, among which light quality and intensity play pivotal roles. Light not only serves as the primary energy source for photosynthesis but also acts as a signal regulating various physiological processes \autocite{eichhorn_bilodeau_update_2019}.

Ultraviolet (UV) light\index{ultraviolet light (UV)}, particularly in the UV-A (\qtyrange[range-phrase=\textendash, range-units=single]{315}{400}{nm}) and UV-B (\qtyrange[range-phrase=\textendash, range-units=single]{280}{315}{nm}) spectra \autocite{international_organization_for_standardization_space_2007}, has been shown to impact plant growth and secondary metabolite\index{metabolite} production. UV-B radiation, despite its relatively low proportion in the solar spectrum, is particularly influential due to its higher energy and potential to cause damage to DNA, proteins, and lipids. However, plants have evolved mechanisms to mitigate these effects and even utilize UV-B as a signal to enhance protective secondary metabolites like flavonoids\index{metabolite!flavonoid} and cannabinoids\index{metabolite!cannabinoid} \autocite{eichhorn_bilodeau_update_2019}.

In cannabis, UV-B exposure has been associated with increased production of Δ9-tetrahydrocannabinol (THC)\index{metabolite!cannabinoid!Δ9-tetrahydrocannabinol (THC)}, a key psychoactive compound, suggesting that UV light might be harnessed to optimize cannabinoid profiles in cultivated drug-type plants \autocite{eichhorn_bilodeau_update_2019, lydon_uv-b_1987}. However, the focus of this experiment was on assessing the impact of UV light on growth parameters\index{growth parameter} such as plant height\index{growth parameter!plant height}, stem diameter\index{growth parameter!stem diameter}, and the number of internodes\index{growth parameter!number of internodes}, rather than on cannabinoid content.

These growth parameters provide a comprehensive evaluation of the overall health, compactness and robustness of a cannabis plant. Plant height is a direct indicator of growth rate and vigor, correlating with overall biomass production. Stem diameter is crucial for plant stability and resistance to environmental stresses, with thicker stems indicating stronger structural integrity and better nutrient transport capabilities. The number of internodes is a key measure of plant development phases, with more internodes typically signifying a bushier plant with potentially more flowering sites, which is desirable for maximizing yield.

This experiment aimed to evaluate the impact of additional UV light on the growth and development of \plant{Cs} seedlings. Cannabis seeds were germinated and grown indoors under LED light\index{LED light} with and without supplementary UV light. The experiment was conducted over 42 days from May 5 to June 15, 2024, with one set of plants receiving standard full-spectrum LED lighting and the other set receiving additional UV light exposure. By comparing these two groups, the study sought to elucidate the effects of UV light on the considered growth parameters, thereby contributing to the optimization of cultivation practices for enhanced growth and development of cannabis plants.