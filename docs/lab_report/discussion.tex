\section{Discussion}

The initial hypothesis of this experiment was that exposure to UV light at controlled low intensities would enhance the germination rate and seedling development of cannabis seeds by inducing protective and growth-promoting biochemical responses. However, the results of this study indicate no significant differences in plant height, stem circumference, or number of internodes between the UV and control groups for the Frisian Dew cultivar.

These findings suggest that the controlled low-intensity UV exposure used in this experiment did not significantly influence the measured growth parameters under the given conditions. This could be due to several factors, including the possibility that the UV exposure level was insufficient to induce noticeable changes in growth parameters.

While the primary focus of this experiment was on growth parameters, it is important to consider other potential benefits of UV exposure that were not directly measured in this study. For example, UV light has been shown to increase the production of secondary metabolites such as cannabinoids and flavonoids, which are important for the plant's defense mechanisms and have significant pharmacological properties. Enhanced production of these compounds could improve the quality and potency of cannabis plants, even if growth parameters remain unchanged.

Additionally, acclimating cannabis seedlings to UV light indoors could have practical benefits for outdoor cultivation. Plants accustomed to UV exposure might be better prepared to handle natural sunlight, potentially leading to improved growth and resilience when transplanted outdoors.

Future experiments could explore these aspects in more detail by including measurements of cannabinoid\index{metabolite!cannabinoid} and flavonoid\index{metabolite!flavonoid} content, as well as examining longer exposure durations and varying intensities of UV light. Moreover, it would be beneficial to study the combined effects of UV light with other environmental factors, such as temperature and humidity, to develop a more comprehensive understanding of how to optimize indoor cultivation conditions for cannabis.

In summary, while the current study did not find significant differences in growth parameters between the UV and control groups, it highlights the need for further research into the broader effects of UV light on cannabis plants. By expanding the scope of future studies, we can better understand how to leverage UV exposure to enhance both the growth and biochemical profiles of cannabis plants.